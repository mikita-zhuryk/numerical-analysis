\documentclass[14pt, a4paper]{article}
\usepackage{fullpage}
\usepackage[top=2cm, bottom=2cm, left=2.5cm, right=2cm]{geometry}
\usepackage{amsmath,amsthm,amsfonts,amssymb,amscd}
\usepackage{fancyhdr}
\usepackage{fixltx2e}
\usepackage{mathrsfs}
\usepackage{listings}
\usepackage{color}
\usepackage{relsize}
\usepackage{graphicx}
\usepackage[utf8]{inputenc}
\usepackage[T1]{fontenc}
\usepackage[english, russian]{babel}

\definecolor{dkgreen}{rgb}{0,0.6,0}
\definecolor{gray}{rgb}{0.5,0.5,0.5}
\definecolor{mauve}{rgb}{0.58,0,0.82}

\DeclareMathSizes{14}{24}{18}{12}

\lstset{frame=tb,
  language=Python,
  aboveskip=3mm,
  belowskip=3mm,
  showstringspaces=false,
  columns=flexible,
  basicstyle={\small\ttfamily},
  numbers=none,
  numberstyle=\tiny\color{gray},
  keywordstyle=\color{blue},
  commentstyle=\color{dkgreen},
  stringstyle=\color{mauve},
  breaklines=true,
  breakatwhitespace=true,
  tabsize=3
}

\renewcommand{\thesection}{\arabic{section}.}
\renewcommand{\thesubsection}{\thesection\arabic{subsection}.}
\renewcommand{\thesubsubsection}{\thesubsection\arabic{subsubsection}.}


\begin{document}
\pagenumbering{gobble}
\begin{titlepage}
\begin{center}
\large{БЕЛОРУССКИЙ ГОСУДАРСТВЕННЫЙ УНИВЕРСИТЕТ 

ФАКУЛЬТЕТ ПРИКЛАДНОЙ МАТЕМАТИКИ И ИНФОРМАТИКИ

КАФЕДРА ВЫЧИСЛИТЕЛЬНОЙ МАТЕМАТИКИ}
\end{center}
\vspace*{\fill}
\begin{center}
Лабораторная работа 3

\large{\textbf{Метод секущих решения нелинейного уравнения}}

Вариант 7
\end{center}
\begin{flushright}
\textbf{Выполнил:}

Журик Никита Сергеевич \\ 2 курс, 6 группа

\textbf{Преподаватель:}

Будник Анатолий Михайлович
\end{flushright}
\vspace*{\fill}
\begin{center}
Минск, 2019
\end{center}
\end{titlepage}

\tableofcontents
\newpage

\newpage
\pagenumbering{arabic}

  \section{Постановка задачи}
    \begin{enumerate}
      \item
      Отделить корень и определить отрезок $[a; b]$.
      \item
      Решить нелинейное уравнение $f(x) = 0$ методом секущих с точностью $\epsilon = 10^{-8}$.
      \item
      Вычислить невязку решения.
      \item
      Проанализировать полученные результаты и сравнить с методом простой итерации и методом Ньютона.
    \end{enumerate}
  \section{Алгоритм решения}
  \begin{itemize}
    \item
    Сперва отделим корни уравнения \begin{equation}f(x) = 0\end{equation} при помощи таблицы значений. Таким образом, для каждого корня получим некоторый отрезок, содержащий сам корень, причём на этом отрезке функция в левой части монотонна.
    \item
    В отличие от метода Ньютона метод секущих не требует нахождения производной функции $f(x)$ в явном виде. Вместо этого используется приближение к производной, так называемая разделённая разность
    \begin{equation}f'(x)|_{x^k} \approx \frac{f(x^k) - f(x^{k-1})}{x^k - x^{k-1}}\end{equation}
    Тогда итерационный процесс для нахождения решения может быть построен следующим образом:
    \begin{equation}x^{k + 1} = x^k - \frac{(x^k - x^{k-1})}{f(x^k) - f(x^{k-1})}f(x)\big|_{x^k}\end{equation}
  \end{itemize}
  \section{Решение конкретного уравнения}
  \begin{itemize}
    \item
    Рассмотрим уравнение $3ln^2x + 6lnx - 5 = 0$ и построим описанный выше итерационный процесс. Указанный итерационный процесс является двухшаговым, поэтому необходимо найти ещё одно приближение к корню перед тем, как начинать итерации.
    \item
    Для этого в качестве начальных приближений к корням используем найденные при решении методом простой итерации, а следующие приближения найдём при помощи одной итерации метода Ньютона.
    Получим: \begin{align*}x_1^0 &= 0.14134905726406316; \\ x_1^1 &= 0.011919072400887187; \\ x_2^0 &= 1.8834601238122997; \\ x_2^1 &= 1.8832389857682699.\end{align*}
  \end{itemize}
  \section{Листинг программы}
  Для реализации алгоритма был использован Python и библиотеки numpy и matplotlib.

\begin{lstlisting}

#Common.py

#!/usr/bin/env python
# coding: utf-8


#Plot the function

import math
import numpy as np
import matplotlib.pyplot as plt

samples = 20
left_border = 0.01
right_border = 3
delta = (right_border - left_border) / samples
eps = 10 ** -8

def f(x):
    l = math.log(x)
    return 3 * l ** 2 + 6 * l - 5

def f_prime(x):
    l = math.log(x)
    return 6 * (l + 1) / x

def phi(x):
    return math.e ** ((-3 * math.log(x) ** 2 + 5) / 6)

def phi_prime(x):
    return -phi(x) * math.log(x) / x

def calc_f_values(x_values, f):
    f_values = np.zeros(np.shape(x_values))
    for i in range(samples):
        f_values[i] = f(x_values[i])
    return f_values

#Root separation

def separate_roots(interval, f, new_samples):
    global left_border
    left_border = interval[0]
    global right_border
    right_border = interval[1]
    global samples
    samples = new_samples
    x_values = np.linspace(left_border, right_border, samples)
    f_values = calc_f_values(x_values, f)
    intervals = np.empty((1, 2))
    for i in range(samples - 1):
        if (f_values[i + 1] * f_values[i] < 0):
            if (intervals.shape[0] == 1):
                intervals = np.array([x_values[i], x_values[i + 1]])
            else:
                intervals = np.vstack((intervals, [x_values[i], x_values[i + 1]]))
    return intervals

def dichotomy(init_intervals):
    
    def dichotomy_single_root(interval):
        if (interval[1] - interval[0] < delta):
            return interval
        print("[{}; {}]".format(interval[0], interval[1]))
        center = (interval[0] + interval[1]) / 2
        print(f(interval[0]) * f(center))
        if (f(interval[0]) * f(center) < 0):
            return dichotomy_single_root([interval[0], center])
        else:
            return dichotomy_single_root([center, interval[1]])
        
    for i in range(len(init_intervals)):
        init_intervals[i] = dichotomy_single_root(init_intervals[i])
    return init_intervals

#Secant.py

#!/usr/bin/env python
# coding: utf-8

from Common import *

def get_intervals(left, right, f, samples):
    intervals = separate_roots((left, right), f, samples)
    return intervals

#Solve the equation using secant method
    
def find_all_roots(intervals, f, epsilon):
    def find_root(interval, f):
        def diff_ratio(old_x, new_x):
            v = (f(new_x) - f(old_x)) / (new_x - old_x) 
            return v
        
        x_left = interval[0]
        x_right = interval[1]
        lam = f(x_left) / f(x_right)
        x_k2 = (x_left - lam * x_right) / (1 - lam)
        x_0 = x_k2
        #old_x = (x_left + x_right) / 2
        x_k1 = x_k2 - f(x_k2) / f_prime(x_k2)
        print("x_1 after one Newton step:", x_k1)
        x_k = x_k1 - f(x_k1) / diff_ratio(x_k2, x_k1)
        iter_num = 2
        while (abs(x_k1 - x_k) >= eps):
            x_k2 = x_k1
            x_k1 = x_k
            x_k = x_k1 - f(x_k1) / diff_ratio(x_k2, x_k1)
            #print(x_k)
            iter_num += 1
        return (x_k1, x_k, iter_num, x_0)
    
    global eps
    eps = epsilon
    roots = np.array([])
    for interval in intervals:
        (x_k, x_k1, iter_num, x_0) = find_root(interval, f)
        print("Interval: [{}; {}]\nx^0 = {}\nRoot: x* = {}; f(x*) = {}\n|x^(k+1) - x^k| = {}\nNumber of iterations: {}\n"
              .format(interval[0], interval[1], x_0, x_k1, f(x_k1), abs(x_k - x_k1), iter_num))
        if len(roots) == 1:
            roots[0] = x_k1
        else:
            roots = np.append(roots, x_k1)
    return roots
    
if __name__ == "__main__":
    intervals = get_intervals(left_border, right_border, f, samples)
    find_all_roots(intervals, f, eps)
\end{lstlisting}

  \section{Вывод программы}
\begin{verbatim}
x_1 after one Newton step: 0.011919072400887187
Interval: [0.01; 0.1673684210526316]
x^0 = 0.14134905726406316
Root: x* = 0.07186304228911596; f(x*) = 1.290345608140342e-11
|x^(k+1) - x^k| = 3.5431435979615955e-09
Number of iterations: 10

x_1 after one Newton step: 1.8832389857682699
Interval: [1.7410526315789476; 1.8984210526315792]
x^0 = 1.8834601238122997
Root: x* = 1.883238990800978; f(x*) = 5.95967719618784e-13
|x^(k+1) - x^k| = 5.032708028096522e-09
Number of iterations: 2
\end{verbatim}

  \section{Выводы}
  \begin{itemize}
  \item
  Как следует из результата выполнения программы, метод секущих сошёлся для меньшего корня, как и метод Ньютона, и сделал это за 10 итераций (одна из них по формуле метода Ньютона) против 8 итераций метода Ньютона.
  \item
  В окрестности второго корня метод сошёлся за две итерации, что совпадает с числом итераций метода Ньютона.
  \item
  Стоит заметить, что невязка как одного решения, так и другого, полученного методом секущих, больше невязки, полученной в методе Ньютона. Число итераций метода секущих также больше, чем число итераций метода Ньютона.
  Оба результата обусловлены тем, что данный метод имеет порядок сходимости $\alpha = \frac{1 + \sqrt{5}}{2} \approx 1.618$ в отличие от второго порядка сходимости метода Ньютона.
  \item
  Ниже приведены результаты лишь для большего корня, так как метод простой итерации не сошёлся к меньшему:
  \begin{align*}
  k_{Secant} &= 2; \\
  k_{Newton} &= 2; \\
  k_{Simple} &= 24; \\ 
  r_{Secant} &= 5.95967719618784e-13; \\
  r_{Newton} &= 0.0; \\
  r_{Simple} &= 1.9699690767538414e-08.
  \end{align*}
  Сравнив результаты, можно отметить, что метод секущих значительно превосходит метод простой итерации как в плане числа итераций, так и в плане невязки решения, однако несколько уступает методу Ньютона в силу меньшей скорости сходимости, что компенсируется более широкой областью применения данного метода.
  В частности, в случае, когда по каким-либо причинам вычислить значение производной не представляется возможным (слишком трудоёмкая процедура по сравнению с вычислением значения самой функции, функция не является непрерывно дифференцируемой на отрезке отделения корня и т.п.), метод Ньютона плохо применим
  в отличие от метода секущих, что является существенным преимуществом данного метода.  
  \end{itemize}

\end{document}