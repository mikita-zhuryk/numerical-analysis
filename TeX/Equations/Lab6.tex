\documentclass[14pt, a4paper]{article}
\usepackage{fullpage}
\usepackage[top=2cm, bottom=2cm, left=2.5cm, right=2cm]{geometry}
\usepackage{amsmath,amsthm,amsfonts,amssymb,amscd}
\usepackage{fancyhdr}
\usepackage{fixltx2e}
\usepackage{mathrsfs}
\usepackage{listings}
\usepackage{color}
\usepackage{relsize}
\usepackage{graphicx}
\usepackage{physics}
\usepackage[utf8]{inputenc}
\usepackage[T1]{fontenc}
\usepackage[english, russian]{babel}

\definecolor{dkgreen}{rgb}{0,0.6,0}
\definecolor{gray}{rgb}{0.5,0.5,0.5}
\definecolor{mauve}{rgb}{0.58,0,0.82}

\DeclareMathSizes{14}{24}{18}{12}

\lstset{frame=tb,
  language=Python,
  aboveskip=3mm,
  belowskip=3mm,
  showstringspaces=false,
  columns=flexible,
  basicstyle={\small\ttfamily},
  numbers=none,
  numberstyle=\tiny\color{gray},
  keywordstyle=\color{blue},
  commentstyle=\color{dkgreen},
  stringstyle=\color{mauve},
  breaklines=true,
  breakatwhitespace=true,
  tabsize=3
}

\renewcommand{\thesection}{\arabic{section}.}
\renewcommand{\thesubsection}{\thesection\arabic{subsection}.}
\renewcommand{\thesubsubsection}{\thesubsection\arabic{subsubsection}.}


\begin{document}
\pagenumbering{gobble}
\begin{titlepage}
\begin{center}
\large{БЕЛОРУССКИЙ ГОСУДАРСТВЕННЫЙ УНИВЕРСИТЕТ 

ФАКУЛЬТЕТ ПРИКЛАДНОЙ МАТЕМАТИКИ И ИНФОРМАТИКИ

КАФЕДРА ВЫЧИСЛИТЕЛЬНОЙ МАТЕМАТИКИ}
\end{center}
\vspace*{\fill}
\begin{center}
Лабораторная работа 6

\large{\textbf{Метод Ньютона решения системы нелинейных уравнений}}

Вариант 7
\end{center}
\begin{flushright}
\textbf{Выполнил:}

Журик Никита Сергеевич \\ 2 курс, 6 группа

\textbf{Преподаватель:}

Будник Анатолий Михайлович
\end{flushright}
\vspace*{\fill}
\begin{center}
Минск, 2019
\end{center}
\end{titlepage}

\tableofcontents
\newpage

\newpage
\pagenumbering{arabic}

  \section{Постановка задачи}
    \begin{enumerate}
      \item
      Отделить корень и определить шар $S_{\delta}$.
      \item
      Решить систему методом Ньютона.
      \item
      Вычислить невязку решения.
      \item
      Проанализировать полученные результаты и сравнить с методом простой итерации и методом Гаусса-Зейделя.
    \end{enumerate}
  \section{Решение системы нелинейных уравнений}
  \begin{itemize}
    \item
    Рассмотрим систему нелинейных уравнений:
    $$\begin{cases}y - \frac{x^2}{2} + x - 0.5 = 0, \\ 2x + y - \frac{y^3}{6} - 1.6 = 0.\end{cases}$$
    Отделим корни путём выражения $y(x)$ из первого уравнения: $y(x) = \frac{x^2}{2} - x + 0.5$. Тогда подставим данное выражение во второе уравнение и отделим корни
    полученного нелинейного уравнения. Получим для каждого корня отрезки $[a_i, b_i]$, положим \begin{equation}x_i^0 = \left(\frac{a_i+b_i}{2}, y\left(\frac{a_i+b_i}{2}\right)\right)^T;
    \ S_{\delta} = \left\{ \bold{x} \bigg| ||\bold{x} - \bold{x}_i^0||_{\infty} \leq \delta, \ \delta = \max \left(\frac{y(a) + y(b)}{2}, \frac{b - a}{2} \right) \right\}\end{equation}
    Как видно из результата выполнения программы, для корней получаем:
    \begin{align*}x_1^0 &= \left(0.78282828, 0.02358178\right)^T \\ r_1 &= 0.05050505 \\ x_2^0 &= \left(3.81313131, 3.95686389\right)^T \\ r_2 &= 0.07103867\end{align*}
    \item
    Построим итерационный процесс для нахождения корней $\bold{x}_i$ уравнения $\bold{F}(\bold{x}) = \bold{0}$:
    \begin{equation}\bold{x}^{k+1} = \bold{x}^k - \left[\frac{d\bold{F}}{d\bold{x}}\right]^{-1}\bold{F}(\bold{x}^k); \bold{x}^0 = \bold{x}_i^0, k = 1, 2, ...\end{equation}
  \end{itemize}
  \section{Листинг программы}
  Для реализации алгоритма был использован Python и библиотеки numpy и matplotlib.

\begin{lstlisting}
#SystemCommon.py
import math
import numpy as np
import matplotlib.pyplot as plt

def f1(x, y):
    return y - 0.5 * x ** 2 + x - 0.5

def f1_xprime(x, y):
    return -x + 1

def f1_yprime(x, y):
    return 1

def f2(x, y):
    return 2 * x + y - (y ** 3) / 6 - 1.6

def f2_xprime(x, y):
    return 2

def f2_yprime(x, y):
    return 1 - (y ** 2) / 2

def f(x, y):
    return np.array([f1(x, y), f2(x, y)], dtype=np.double)

def df(x, y):
    return np.array([[f1_xprime(x, y), f1_yprime(x, y)], [f2_xprime(x, y), f2_yprime(x, y)]], dtype=np.double)

def ysub(x):
    return 0.5 * x ** 2 - x + 0.5

def ysub_prime(x):
    return x - 1

def f2_ysub(x):
    return f2(x, ysub(x))

def f2_ysub_prime(x):
    return 2 + ysub_prime(x) - (ysub(x) ** 2) * ysub_prime(x) / 2

def infNorm(x):
    return np.max(np.abs(x))

import Newton

intervals = Newton.get_intervals_table(0.0, 5.0, f2_ysub, 100)

roots = np.array([(interval[0] + interval[1]) / 2 for interval in intervals])
radii = np.array([max((ysub(interval[1]) - ysub(interval[0])) / 2, interval[1] - interval[0]) for interval in intervals])

print("x values for starting points: " + str(roots))
print("Sphere radia: " + str(radii))

points = np.array([(root, ysub(root)) for root in roots], dtype=np.double)

print("Starting points: " + str(points))

#SystemNewton.py

from SystemCommon import *

def systemNewton(f, df, point, outerEps):
    def invert(A):
        return np.linalg.inv(A)
    
    prevPoint = np.copy(point)
    jacobi = df(prevPoint[0], prevPoint[1])
    point = prevPoint - np.dot(invert(jacobi), f(prevPoint[0], prevPoint[1]))
    iterNum = 1
    while (infNorm(point - prevPoint) >= outerEps):
        prevPoint = np.copy(point)
        jacobi = df(prevPoint[0], prevPoint[1])
        point = prevPoint - np.dot(invert(jacobi), f(prevPoint[0], prevPoint[1]))
        iterNum += 1
    return (point, iterNum)
        
if __name__ == '__main__':
    roots = []

    for point in points:
        print("Starting point: " + str(point))
        print("Deficiency before: " + str(f(point[0], point[1])))
        (root, iterNum) = systemNewton(f, df, point, 10 ** -5)
        print("Newton root: " + str(root))
        if len(roots) == 0:
            roots = [root]
        else:
            roots.append(root)
        print("Deficiency after: " + str(f(root[0], root[1])))
        print("Number of iterations: " + str(iterNum))
        print("\n")
\end{lstlisting}

  \section{Вывод программы}
\begin{verbatim}
x values for starting points: [0.78282828 3.81313131]
Sphere radia: [0.05050505 0.07103867]
Starting points: [[0.78282828 0.02358178]
 [3.81313131 3.95685389]]

Starting point: [0.78282828 0.02358178]
Deficiency before: [ 0.         -0.01076384]
(0.788855413987323395, 0.022291018049479104)
(0.788855413960984908, 0.022291018106793549)
Newton root: [0.78885541 0.02229102]
Deficiency after: [0. 0.]
Number of iterations: 3


Starting point: [3.81313131 3.95685389]
Deficiency before: [ 0.         -0.34209107]
(3.792799298934621532, 3.899863859260823240)
(3.792799122123207578, 3.899863468266064004)
Newton root: [3.79279912 3.89986347]
Deficiency after: [-1.55431223e-14 -2.97095681e-13]
Number of iterations: 3
\end{verbatim}

  \section{Выводы}
  \begin{itemize}
  \item
  В отличие от метода Гаусса-Зейделя, метод Ньютона сошёлся в окрестности меньшего корня за 3 итерации с невязкой, равной нулю.
  \item
  Сравним точность и скорость сходимости рассмотренных методов в окрестности большего корня:
  \begin{align*}
  k_{Gauss-Zeidel} &= 9; \\
  k_{Newton} &= 3; \\
  k_{SimpleIter} &= 4; \\
  ||r_{Gauss_Zeidel}||_{\infty} &= 6.98076505e-06; \\
  ||r_{Newton}||_{\infty} &= 2.97095681e-13; \\
  ||r_{SimpleIter}||_{\infty} &= 3.22813787e-07.
  \end{align*}
  \item
  Таким образом, метод Ньютона обладает самой высокой скоростью сходимости и, как следствие, точностью среди рассмотренных методов решения систем нелинейных уравнений. Однако, как и в случае одного уравнения, он предполагает ограничения на уравнения системы: в частности, каждая функция $f_i(\bold{x})$ должна быть непрерывно дифференцируемой по всем переменным, что сужает область применения данного метода. Данное замечание справедливо и для рассмотренного ранее метода простой итерации в силу использования частных производных функций системы для нахождения канонического вида, в то время, как рассмотренная реализация метода Гаусса-Зейделя требует лишь непрерывных частных производных $\pdv{f_i}{x_i}$.
  \end{itemize}

\end{document}