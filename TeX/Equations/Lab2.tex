\documentclass[14pt, a4paper]{article}
\usepackage{fullpage}
\usepackage[top=2cm, bottom=2cm, left=2.5cm, right=2cm]{geometry}
\usepackage{amsmath,amsthm,amsfonts,amssymb,amscd}
\usepackage{fancyhdr}
\usepackage{fixltx2e}
\usepackage{mathrsfs}
\usepackage{listings}
\usepackage{color}
\usepackage{relsize}
\usepackage{graphicx}
\usepackage[utf8]{inputenc}
\usepackage[T1]{fontenc}
\usepackage[english, russian]{babel}

\definecolor{dkgreen}{rgb}{0,0.6,0}
\definecolor{gray}{rgb}{0.5,0.5,0.5}
\definecolor{mauve}{rgb}{0.58,0,0.82}

\DeclareMathSizes{14}{24}{18}{12}

\lstset{frame=tb,
  language=Python,
  aboveskip=3mm,
  belowskip=3mm,
  showstringspaces=false,
  columns=flexible,
  basicstyle={\small\ttfamily},
  numbers=none,
  numberstyle=\tiny\color{gray},
  keywordstyle=\color{blue},
  commentstyle=\color{dkgreen},
  stringstyle=\color{mauve},
  breaklines=true,
  breakatwhitespace=true,
  tabsize=3
}

\renewcommand{\thesection}{\arabic{section}.}
\renewcommand{\thesubsection}{\thesection\arabic{subsection}.}
\renewcommand{\thesubsubsection}{\thesubsection\arabic{subsubsection}.}


\begin{document}
\pagenumbering{gobble}
\begin{titlepage}
\begin{center}
\large{БЕЛОРУССКИЙ ГОСУДАРСТВЕННЫЙ УНИВЕРСИТЕТ 

ФАКУЛЬТЕТ ПРИКЛАДНОЙ МАТЕМАТИКИ И ИНФОРМАТИКИ

КАФЕДРА ВЫЧИСЛИТЕЛЬНОЙ МАТЕМАТИКИ}
\end{center}
\vspace*{\fill}
\begin{center}
Лабораторная работа 2

\large{\textbf{Метод Ньютона решения нелинейного уравнения}}

Вариант 7
\end{center}
\begin{flushright}
\textbf{Выполнил:}

Журик Никита Сергеевич \\ 2 курс, 6 группа

\textbf{Преподаватель:}

Будник Анатолий Михайлович
\end{flushright}
\vspace*{\fill}
\begin{center}
Минск, 2019
\end{center}
\end{titlepage}

\tableofcontents
\newpage

\newpage
\pagenumbering{arabic}

  \section{Постановка задачи}
    \begin{enumerate}
      \item
      Отделить корень и определить отрезок $[a; b]$.
      \item
      Проверить условия теоремы о сходимости метода Ньютона.
      \item
      Решить нелинейное уравнение $f(x) = 0$ методом Ньютона с точностью $\epsilon = 10^{-8}$.
      \item
      Найти априорную и фактическую оценки количества итераций.
      \item
      Вычислить невязку решения.
      \item
      Проанализировать полученные результаты и сравнить с методом простой итерации.
    \end{enumerate}
  \section{Алгоритм решения}
  \begin{itemize}
    \item
    Сперва отделим корни уравнения \begin{equation}f(x) = 0\end{equation} при помощи таблицы значений. Таким образом, для каждого корня получим некоторый отрезок, содержащий сам корень, причём на этом отрезке функция в левой части монотонна.
    \item
    В отличие от метода простой итерации метод Ньютона не требует представления уравнения в каноническом виде. Однако при этом он накладывает на функцию исходного уравнения более строгое ограничение, а именно,
    функция в левой части исходного уравнения должна быть дифференцируема. Тогда итерационный процесс для нахождения решения может быть построен следующим образом:
    \begin{equation}x^{k + 1} = x^k - \frac{f(x)}{f'(x)}\bigg|_{x^k}\end{equation}
    \item
    Справделива следующая теорема: \\
    \textbf{Теорема о сходимости метода Ньютона.} \\
    Пусть 
    \begin{itemize}
    \item
    $f(x) \in C^{(2)}(S_0)$, где $S_0 = [x^0; x^0 + 2h_0]; h_0 = -\frac{f(x)}{f'(x)}\bigg|_{x^0}$, причём на концах отрезка $S_0$ $f(x) \neq 0$ и $f'(x) \neq 0$;
    \item
    Для начального приближения справедливо $2|h_0|M \leq |f'(x^0)|$, где $M = \max\limits_{x \in S_0} |f''(x)|$.
    \end{itemize}
    Тогда:
    \begin{itemize}
    \item
    Внутри $S_0$ лежит единственный корень уравнения $(1)$;
    \item
    Последовательность приближений $x^k$ может быть построена по $x^0$ с использованием $(2)$;
    \item
    Последовательность $x^k$ сходится к корню $x^*$;
    \item
    Скорость сходимости характеризуется неравенством:
    \begin{equation}|x^* - x^{k + 1}| \leq |x^{k + 1} - x^k| \leq \frac{M}{2|f'(x^k)|}|x^k - x^{k - 1}|^2, k = 1, 2, ... \end{equation}
    \end{itemize}
    Из последнего неравенства следует априорная оценка числа итераций:
    \begin{equation} k \geq log_2\frac{ln(\alpha\epsilon)}{ln(\alpha|x^1 - x^0|)}, \alpha = \max\limits_{x \in S_0} \left|\frac{f''(x)}{2f'(x)}\right| \end{equation}
  \end{itemize}
  \section*{Решение конкретного уравнения}
  \begin{itemize}
    \item
    Рассмотрим уравнение $3ln^2x + 6lnx - 5 = 0$. Вычислим производную функции $f(x)$ в явном виде: \begin{equation}f'(x) = \frac{6lnx}{x} + \frac{6}{x}\end{equation}
    \item
    Тогда итерационный процесс будет выглядеть следующим образом: \begin{equation}x^{k + 1} = x^k - \frac{3ln^2x + 6lnx - 5}{\frac{6}{x}(lnx + 1)}\bigg|_{x^k}; k = 0, 1, 2, ...; x^0\end{equation}
    \item
    В качестве начальных приближений к корням используем найденные при решении методом простой итерации отрезки.
    \item
    Проверим условия теоремы о сходимости метода Ньютона.
    \begin{align*} f''(x) &= -6\frac{lnx}{x^2} \\ f'''(x) &= \frac{12lnx - 6}{x^3} \\ f'''(x) = 0 &\iff lnx = \frac{1}{2} \implies x = 1.6487212707\end{align*}
    Так как функция трижды дифференцируема на $(0; +\infty)$, она дважды непрерывно дифференцируема на любом отрезке, вложенном в него.
    \begin{enumerate}
    \item
    Рассмотрим отрезок, содержащий меньший корень $[0.01; 0.1673684210526316]$.
    \begin{equation*}h_0 = -0.12942998486317597\end{equation*}
    Так как $x^0 + 2h_0 < 0$, $f(x)$ не определена на $S_0$. Следовательно, условия теоремы не выполняются для меньшего корня.
    
    \item
    Теперь рассмотрим отрезок $[1.7410526315789476; 1.8984210526315792]$, содержащий больший корень:
    \begin{equation*}h_0 = -0.00022113804402974617\end{equation*}
    Проверим второе условие: \begin{align*}f'''(x) &\neq 0 \ \forall x \in S_0 \implies \\ M &= \max\{|f''(1.88301784772424)|, |f''(1.8834601238122997)|\} \\ &= 1.0709293413775358 \\
    2|h_0|M &= 0.00047364643969258517 \\ |f'(x^0)| &= 5.202479912227185\end{align*}
    Легко видеть, что и второе условие выполнено в окрестности большего корня. Таким образом, для него справедлива теорема и возможна априорная оценка числа итераций:
    \begin{equation}log_2\frac{ln(\alpha\epsilon)}{ln(\alpha|x^1 - x^0|)} = \left[\begin{aligned}\frac{f''(x)}{f'(x)} &= -\frac{lnx}{x(lnx+1)} \implies \\
    \alpha &= \max\limits_{x \in S_0} \left|\frac{f''(x)}{f'(x)}\right| \\ &= 0.2058310407946963193 \end{aligned} \right] = 1.0004713925101845\end{equation}
    Тогда \begin{equation}k_{aprior} = 2\end{equation}

    \end{enumerate}
  \end{itemize}
  \section{Листинг программы}
  Для реализации алгоритма был использован Python и библиотеки numpy и matplotlib.

\begin{lstlisting}

#Common.py

#!/usr/bin/env python
# coding: utf-8

#Plot the function

import math
import numpy as np
import matplotlib.pyplot as plt

samples = 20
left_border = 0.01
right_border = 3
delta = (right_border - left_border) / samples
eps = 10 ** -5

def f(x):
    l = math.log(x)
    return 3 * l ** 2 + 6 * l - 5

def f_prime(x):
    l = math.log(x)
    return 6 * (l + 1) / x

def phi(x):
    return math.e ** ((-3 * math.log(x) ** 2 + 5) / 6)

def phi_prime(x):
    return -phi(x) * math.log(x) / x

x_values = np.linspace(left_border, right_border, samples)
f_values = np.zeros(np.shape(x_values))

for i in range(samples):
    f_values[i] = f(x_values[i])

#Root separation

def separate_roots(x_values, f_values):
    intervals = [ ]
    for i in range(samples - 1):
        if (f_values[i + 1] * f_values[i] < 0):
            intervals = np.append(intervals, i)
    return intervals

def dichotomy(init_intervals):
    
    def dichotomy_single_root(interval):
        if (interval[1] - interval[0] < delta):
            return interval
        center = (interval[0] + interval[1]) / 2
        if (f(interval[0]) * f(center) < 0):
            return dichotomy_single_root([interval[0], center])
        else:
            return dichotomy_single_root([center, interval[1]])
        
    for i in range(len(init_intervals)):
        init_intervals[i] = dichotomy_single_root(init_intervals[i])
    return init_intervals

#Newton.py

#!/usr/bin/env python
# coding: utf-8

from Common import *

intervals = separate_roots(x_values, f_values)
print(intervals)

#Solve the equation using Newton method

def find_root(interval):
    x_left = interval[0]
    x_right = interval[1]
    lam = f(x_left) / f(x_right)
    old_x = (x_left - lam * x_right) / (1 - lam)
    new_x = old_x - f(old_x) / f_prime(old_x)
    iter_num = 1
    while (abs(old_x - new_x) >= eps):
        old_x = new_x
        new_x = old_x - f(old_x) / f_prime(old_x)
        iter_num += 1
    return (old_x, new_x, iter_num)

for interval in intervals:
    (x_k, x_k1, iter_num) = find_root(interval)
    print("Interval: [{}; {}]\nx^0 = {}\nRoot: x* = {}; f(x*) = {}\n|x^(k+1) - x^k| = {}\nNumber of iterations: {}\n"
          .format(interval[0], interval[1], x_0, x_k1, f(x_k1), abs(x_k - x_k1), iter_num))
\end{lstlisting}

  \section{Вывод программы}
\begin{verbatim}
Interval: [0.01; 0.1673684210526316]
x^0 = 0.14134905726406316
Root: x* = 0.07186304228921059; f(x*) = 3.552713678800501e-15
|x^(k+1) - x^k| = 2.6333352165508472e-11
Number of iterations: 8

Interval: [1.7410526315789476; 1.8984210526315792]
x^0 = 1.8834601238122997
Root: x* = 1.8832389908008633; f(x*) = 0.0
|x^(k+1) - x^k| = 5.032593453080381e-09
Number of iterations: 2


\end{verbatim}

  \section{Выводы}
  \begin{itemize}
  \item
  Как следует из результата выполнения программы, метод сошёлся для меньшего корня к решению уравнения, несмотря на то, что условия теоремы не были выполнены.
  Из числа итераций видно, что сходился метод достаточно медленно (по сравнению со вторым корнем), но требуемая точность была достигнута.
  \item
  В окрестности второго корня метод сошёлся за две итерации, что совпадает с априорной оценкой в силу того, что оценённое количество итераций мало. Выбранная точность $\epsilon = 10^{-8}$
  привела к тому, что невязка решения оказалась равной нулю.
  \item
  Ниже приведены результаты лишь для большего корня, так как метод простой итерации не сошёлся к меньшему:
  \begin{align*}
  k_{Newton} &= 2 \\
  k_{Simple} &= 24 \\ 
  r_{Newton} &= 0.0 \\
  r_{Simple} &= 1.9699690767538414e-08
  \end{align*}
  Сравнивая метод Ньютона с методом простой итерации, можно заметить, что метод Ньютона сходится к решению уравнения значительно быстрее, а само приближение к решению точнее, чем в методе простой итерации.
  Это обусловлено квадратичной скоростью сходимости метода Ньютона в сравнении с линейной скоростью сходимости метода простой итерации, что делает его очень полезным при решении уравнений с непрерывно дифференцируемой функцией.
  \end{itemize}

\end{document}