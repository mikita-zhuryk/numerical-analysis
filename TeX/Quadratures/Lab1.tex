\documentclass[14pt, a4paper]{article}
\usepackage{fullpage}
\usepackage[top=2cm, bottom=2cm, left=2.5cm, right=2cm]{geometry}
\usepackage{amsmath,amsthm,amsfonts,amssymb,amscd}
\usepackage{fancyhdr}
\usepackage{fixltx2e}
\usepackage{mathrsfs}
\usepackage{listings}
\usepackage{color}
\usepackage{relsize}
\usepackage{graphicx}
\usepackage[utf8]{inputenc}
\usepackage[T1]{fontenc}
\usepackage[english, russian]{babel}

\definecolor{dkgreen}{rgb}{0,0.6,0}
\definecolor{gray}{rgb}{0.5,0.5,0.5}
\definecolor{mauve}{rgb}{0.58,0,0.82}

\DeclareMathSizes{14}{24}{18}{12}

\lstset{frame=tb,
  language=Python,
  aboveskip=3mm,
  belowskip=3mm,
  showstringspaces=false,
  columns=flexible,
  basicstyle={\small\ttfamily},
  numbers=none,
  numberstyle=\tiny\color{gray},
  keywordstyle=\color{blue},
  commentstyle=\color{dkgreen},
  stringstyle=\color{mauve},
  breaklines=true,
  breakatwhitespace=true,
  tabsize=3
}

\renewcommand{\thesection}{\arabic{section}.}
\renewcommand{\thesubsection}{\thesection\arabic{subsection}.}
\renewcommand{\thesubsubsection}{\thesubsection\arabic{subsubsection}.}

\title {Лабораторная работа №1 \\ Решение нелинейного уравнения методом простой итерации}
\author {Журик Никита, 6 группа}
\date {26.03.2019}


\begin{document}
\pagenumbering{gobble}
\begin{titlepage}
\begin{center}
\large{БЕЛОРУССКИЙ ГОСУДАРСТВЕННЫЙ УНИВЕРСИТЕТ 

ФАКУЛЬТЕТ ПРИКЛАДНОЙ МАТЕМАТИКИ И ИНФОРМАТИКИ

КАФЕДРА ВЫЧИСЛИТЕЛЬНОЙ МАТЕМАТИКИ}
\end{center}
\vspace*{\fill}
\begin{center}
Лабораторная работа 1

\large{\textbf{Применение интерполяционных квадратурных формул для вычисления определённых интегралов.}}

Вариант 5
\end{center}
\begin{flushright}
\textbf{Выполнил:}

Журик Никита Сергеевич \\ 2 курс, 6 группа

\textbf{Преподаватель:}

Будник Анатолий Михайлович
\end{flushright}
\vspace*{\fill}
\begin{center}
Минск, 2019
\end{center}
\end{titlepage}

\tableofcontents
\newpage

\newpage
\pagenumbering{arabic}

  \section{Постановка задачи}
    \begin{enumerate}
      \item
	При помощи правила Рунге вычислить интеграл и определить шаг, необходимый для достижения требуемой точности;
	\item
	Пользуясь формулами для оценки погрешности КФ средних прямоугольников и Симпсона определить необходимые шаги $h_r, h_S$ для достижения требуемой точности;
	\item
	Применить НАСТ Гаусса и оценить погрешность интегрирования;
	\item
	Сравнить полученные результаты.
    \end{enumerate}
  \section{Задание 1.}
  \begin{itemize}
	\item
	Для вычисления интеграла с требуемой точностью воспользуемся следующим критерием остановки итерационного процесса: $$R(f) = \frac{|I_{h_1}-I_{h_2}|}{1-\left(\frac{h_2}{h_1}\right)^2} \leq \epsilon$$
	На каждой итерации будем уменьшать шаги в два раза до достижения требуемой точности. Тогда искомый шаг - $h_2$ на последней итерации.
	Для улучшения точности положим $$I(f) \approx I_{h_2} + R(f)$$
  \end{itemize}
  \section{Листинг программы}
  Для реализации алгоритма был использован Python и библиотеки numpy и scipy.

\begin{lstlisting}

\end{lstlisting}

  \section{Вывод программы}
\begin{verbatim}
\end{verbatim}

  \section{Выводы}

\end{document}