\documentclass[14pt, a4paper]{article}
\usepackage{fullpage}
\usepackage[top=2cm, bottom=2cm, left=2.5cm, right=2cm]{geometry}
\usepackage{amsmath,amsthm,amsfonts,amssymb,amscd}
\usepackage{fancyhdr}
\usepackage{fixltx2e}
\usepackage{mathrsfs}
\usepackage{listings}
\usepackage{color}
\usepackage{relsize}
\usepackage{graphicx}
\usepackage[utf8]{inputenc}
\usepackage[T1]{fontenc}
\usepackage[english, russian]{babel}

\definecolor{dkgreen}{rgb}{0,0.6,0}
\definecolor{gray}{rgb}{0.5,0.5,0.5}
\definecolor{mauve}{rgb}{0.58,0,0.82}

\DeclareMathSizes{14}{24}{18}{12}

\lstset{frame=tb,
  language=Python,
  aboveskip=3mm,
  belowskip=3mm,
  showstringspaces=false,
  columns=flexible,
  basicstyle={\small\ttfamily},
  numbers=none,
  numberstyle=\tiny\color{gray},
  keywordstyle=\color{blue},
  commentstyle=\color{dkgreen},
  stringstyle=\color{mauve},
  breaklines=true,
  breakatwhitespace=true,
  tabsize=3
}

\renewcommand{\thesection}{\arabic{section}.}
\renewcommand{\thesubsection}{\thesection\arabic{subsection}.}
\renewcommand{\thesubsubsection}{\thesubsection\arabic{subsubsection}.}

\begin{document}
\pagenumbering{gobble}
\begin{titlepage}
\begin{center}
\large{БЕЛОРУССКИЙ ГОСУДАРСТВЕННЫЙ УНИВЕРСИТЕТ 

ФАКУЛЬТЕТ ПРИКЛАДНОЙ МАТЕМАТИКИ И ИНФОРМАТИКИ

КАФЕДРА ВЫЧИСЛИТЕЛЬНОЙ МАТЕМАТИКИ}
\end{center}
\vspace*{\fill}
\begin{center}
Лабораторная работа 12

\large{\textbf{Интерполирование функции с использованием кубического сплайна}}

Вариант 7
\end{center}
\begin{flushright}
\textbf{Выполнил:}

Журик Никита Сергеевич \\ 2 курс, 6 группа

\textbf{Преподаватель:}

Будник Анатолий Михайлович
\end{flushright}
\vspace*{\fill}
\begin{center}
Минск, 2019
\end{center}
\end{titlepage}

\tableofcontents
\newpage

\newpage
\pagenumbering{arabic}

  \section{Постановка задачи}
    \begin{enumerate}
      \item
      Интерполировать исходную функцию кубическим сплайном;
      \item
      Вычислить теоретическую оценку и действительную невязку интерполирования;
      \item
      Проанализировать результаты и сравнить с методом интерполирования по равномерной сетке.
    \end{enumerate}
  \section{Алгоритм решения}
  \begin{itemize}
     \item
     Рассмотрим задачу интерполирования исходной функции $f(x)$ при помощи естественного кубического сплайна на равномерной сетке узлов. Отсюда $M_0 = M_N = 0; h_i = \delta = 0.1, i = \overline{1, N}; N = 10$.
     \item 
     Кубический сплайн может быть построен по формуле: \begin{equation}S_3(x) = \frac{M_{i-1}}{6\delta}(x_i - x)^3 + \frac{M_i}{6\delta}(x - x_{i - 1})^3 + \frac{(x - x_{i - 1})}{\delta}\left(f_i - M_i\frac{\delta^2}{6}\right) + \frac{x_i - x}{6}\left(f_{i - 1} - M_{i - 1}\frac{\delta^2}{6}\right)\end{equation}
     \item
     Для нахождения моментов $M_i$ рассмотрим следующую СЛАУ: \begin{equation}\frac{\delta}{6}M_{i - 1} + \frac{2\delta}{3}M_i + \frac{\delta}{6}M_{i + 1} = \frac{f_{i + 1} - f_i}{\delta} - \frac{f_i - f_{i - 1}}{\delta}, i = \overline{1, N - 1}\end{equation}
     Решив данную СЛАУ, построим искомый кубический сплайн.
     \item
     Невязку оценим по формуле \begin{equation}|r_3(x)| \leq \delta^4\max\limits_{x \in [a, b]} |f^{(4)}(x)|\end{equation}
  \end{itemize}
  \section{Листинг программы}
  Для реализации алгоритма был использован Python и библиотеки numpy и matplotlib.

\begin{lstlisting}
#CubicSpline.py

import numpy as np
from math import exp, log, factorial
import matplotlib.pyplot as plt

a = 1.0
b = 2.0
N = 10
delta = (b - a) / N
alpha = 1.7

points = [a + i * delta for i in range(N + 1)]

def f(x):
    return alpha * exp(-x) + (1 - alpha) * log(x)

def fDeriv4(x):
    return alpha * exp(-x) - (1 - alpha) * factorial(3) / x ** 4

def maxDeriv4(samples):
    space = np.linspace(a, b, samples)
    return np.max(np.abs(np.array([(fDeriv4(x)) for x in space], dtype=np.double)))

def S3(M, k):
    return lambda x: (M[k - 1] * (points[k] - x) ** 3 / (6 * delta) + M[k] * (x - points[k - 1]) ** 3 / (6 * delta) 
    + (x - points[k - 1]) * (f(points[k]) - M[k] * delta ** 2 / 6) / delta
    + (points[k] - x) * (f(points[k - 1]) - M[k - 1] * delta ** 2 / 6) / delta)
        
def findMoments():
    A = np.zeros((N + 1, N + 1), dtype = np.double)
    b = np.zeros(N + 1, dtype = np.double)
    A[0][0] = A[N][N] = 1
    b[0] = b[N] = 0
    for i in range(1, N):
        A[i][i - 1] = delta / 6
        A[i][i] = 2 * delta / 3
        A[i][i + 1] = delta / 6
        b[i] = (f(points[i + 1]) - f(points[i])) / delta - (f(points[i]) - f(points[i - 1])) / delta
    return np.linalg.solve(A, b)

def deficiency():
    return maxDeriv4(10000) * delta ** 3

if __name__ == "__main__":
    M = findMoments()
    
    check = [points[0] + delta / 2.6]
    
    [print("S3({0}) = {1}".format(x, S3(M, 1)(x))) for x in check]
    print()
    
    [print("r3({0}) = {1}".format(x, f(x) - S3(M, 1)(x))) for x in check]
    print()
    
    print("Expected deficiency on whole interval: " + str(deficiency()))
    
    print("Real deficiency in control points: " + 
          str(np.max(np.abs(np.array([(f(x) - S3(M, 1)(x)) for x in check], dtype=np.double)))))
\end{lstlisting}

  \section{Вывод программы}
\begin{verbatim}
S3(1.0384615384615385) = 0.5760276330461487

r3(1.0384615384615385) = -0.0006477719097222057

Expected deficiency on whole interval: 0.0048253950499914525
Real deficiency in control points: 0.0006477719097222057

\end{verbatim}

  \section{Выводы}
  \begin{itemize}
  \item
  В результате применения формул для равномерной сетки узлов удалось интерполировать исходную функцию многочленом третьей степени с точностью $r_{EqSp_{control}} = 1.328510657672144e-05$ (в рассматриваемой точке $x^*$). При помощи кубического сплайна была достигнута точность $r_{Cube_{control}} = 6.477719097222057e-04$.
  \item
  Таким образом, погрешность интерполирования при использовании кубического сплайна больше. Преимуществом данного метода по сравнению с рассмотренным ранее является отсутствие необходимости рассматривать различные случаи местоположения точки интерполирования, что делает интерполирование при помощи кубического сплайна более универсальным методом.
  \end{itemize}

\end{document}